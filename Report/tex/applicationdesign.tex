\chapter{Application Design}
\label{chap:design}
In this chapter, the considerations made when designing the system are outlined. Additionally, the metamodel for the survey specification language is presented.

\section{Design Considerations}
\label{sec:designconsiderations}
%TODO intro

\subsection{Survey Navigation}
\label{subsec:surveynavigation}
%TODO section intro
%TODO this section needs a lot more written in it
When running a survey on the phone, the user will expect the questions to be stored intermediately, allowing him/her to navigate back and forth between questions without discarding the answers already typed in.
\begin{figure}[!htbp]
 \centering
 \begin{subfigure}{.45\textwidth}
    \centering
    \begin{tikzpicture}[->,>=stealth',shorten >=0.8pt,auto,node distance=2.2cm,
  thick,main node/.style={circle,fill=blue!20,draw,font=\sffamily\bfseries}]

  \node[main node] (1) {$q_1$};
  \node[main node] (2) [below left of=1] {$q_2$};
  \node[main node] (3) [below left of=2] {$q_3$};
  \node[main node] (4) [below right of=2] {$q_4$};
  \node[main node] (5) [below of=4] {$q_5$};  
  \node[main node] (6) [below right of=1] {$q_6$};
  
  \path[every node/.style={font=\sffamily\small}]
    (1) edge node [right] {male} (6)
        edge node [left] {female} (2)
    (2) edge node [right] {single} (4)
        edge node [left] {married} (3)
    (4) edge node []  {} (5);
    \end{tikzpicture}
    \caption{Question graph}
    \label{sfig:question-graph}
  \end{subfigure}
  \begin{subfigure}{.45\textwidth}
  \begin{tikzpicture}[->,>=stealth',shorten >=0.8pt,auto,node distance=2.2cm,
  thick,
  visited node/.style={circle,fill=blue!20,draw,font=\sffamily\bfseries},
  not visited node/.style={circle,fill=orange!20,draw,font=\sffamily\bfseries},
  chosen node/.style={circle,fill=green!20,draw,font=\sffamily\bfseries}]


  \node[chosen node] (1) {$a_1$};
  \node[chosen node] (2) [below left of=1] {$a_2$};
  \node[visited node] (3) [below left of=2] {$a_3$};
  \node[chosen node] (4) [below right of=2] {$a_4$};
  \node[chosen node] (5) [below of=4] {$a_5$};  
  \node[not visited node] (6) [below right of=1] {$a_6$};

  \path[every node/.style={font=\sffamily\small}]
    (1) edge [dashed] node [right, dashed] {male} (6)
        edge node [left] {female} (2)
    (2) edge node [right] {single} (4)
        edge node [left] {married} (3)
    (4) edge node []  {} (5);
   \end{tikzpicture}
   \caption{Answer graph}
    \label{sfig:answer-graph}
  \end{subfigure}
  \caption{Graph of survey questions and answers, and the relation between them.}
  \label{fig:application graphs}
\end{figure}

Figure \ref{fig:application graphs}, illustrates the design view of a survey. We basically regard the survey as a graph containing the questions. If there is a condition on the next question, the graph branches. In figure \ref{sfig:question-graph} the initial question is regarding the sex of respondent, and there are only additional questions for females. In the application design we mirror our questions in an answer graph, storing answers along with the path taken through the graph. Once the survey is done, the answers are harvested from the answer graph by traversing backwards through the graph. This design allows the user to go back and change any answer without losing information previously entered. We only store answers that are actually answered (questions that are visited by the respondent). An example of this is illustrated in figure \ref{sfig:answer-graph}, where the respondent starts by choosing ``female'' and ``married'' and starts filling out question $q_5$. For some reason she has forgotten that she is not married and goes back to $q_2$ to correct this. She then chooses ``single'' and answers $q_4$ and $q_5$ and completes the survey. Figure \ref{sfig:answer-graph} depicts the path taken (green nodes), the answered but not taken path (blue), and unvisited path (orange).

\subsection{Survey Modularity and Reuse}
\label{subsec:modularityandreuse}
%TODO intro
%TODO write about survey modularity, question references, scheme/instance etc.
%TODO mention the DSL for a smooth transition to next section

\section{Metamodel}
\label{sec:metamodel}
This section presents the metamodel for the survey domain specific language.  

The model shown in figure \ref{fig:metamodel} supports all the examples presented in chapter \ref{chap:examples} (sometimes with small extensions, e.g. to DataType class).
The model consists of two main parts - Survey Scheme, which defines the DSL structure and Survey Instance(s).

\begin{figure}[!htbp]
  \includegraphics[scale=0.55]{\imgdir meta-model}
  \caption{Metamodel for the survey DSL.}
  \label{fig:metamodel}
\end{figure}

\subsection{Survey Scheme}
\label{subsec:surveyscheme}
In the scheme, the structure of a survey is depicted. One of the highlights is usage of question references. Doing so makes it much easier to reuse questions across multiple survey schemes - for example, one could envision creating a question database, whereto the references would point.
A survey is built up modularly of survey parts, which in turn contain references to questions. Optionally, both survey parts and questions can have prerequisites. Adding prerequisites to questions is merely a matter of convenience though, as conceptually, one could do with only adding prerequisites to survey parts. A drawback of that approach, however, would be that a survey part would have to be created every time one would want to add a prerequisite to a single question.
There is nothing surprising about the structure of a question - it has some text, a number of possible answer options, a type (single choice, multiple choice, free text) and a data type (purely for the benefit of the researchers, as it might ease result processing).

\subsection{Survey Instance}
\label{subsec:surveyinstance}
The key idea behind differentiating between survey scheme and instance is, that a survey scheme could be stored centrally (e.g. on a server) and then distributed to all potential respondents, whereupon individual survey instances would be created on each individual smartphone. The individual user answers would be stored in a survey instance along with the users personal data and could then be sent back to some central repository. Distinguishing between multiple instances of a survey on an individual users smartphone would also be possible, thus conveniently enabling the survey recurrence feature.