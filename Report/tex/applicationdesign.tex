\chapter{Application Design}
\label{chap:design}
In this chapter, the considerations made when designing the system are outlined. Additionally, the metamodel for the survey specification language is presented.

\section{Metamodel}
\label{sec:metamodel}
This section presents the metamodel for the survey domain specific language.  

The model shown in figure \ref{fig:metamodel} supports all the examples presented in chapter \ref{chap:examples} (sometimes with small extensions, e.g. to DataType class).
The model consists of two main parts - Survey Scheme, which defines the DSL structure and Survey Instance(s).

\begin{figure}[!ht]
  \includegraphics[scale=0.48]{\imgdir "meta-model"}
  \caption{Metamodel for the survey DSL.}
  \label{fig:metamodel}
\end{figure}

\subsection{Survey Scheme}
\label{subsec:surveyscheme}
In the scheme, the structure of a survey is depicted. One of the highlights is usage of question references. Doing so makes it much easier to reuse questions across multiple survey schemes - for example, one could envision creating a question database, whereto the references would point.
A survey is built up modularly of survey parts, which in turn contain references to questions. Optionally, both survey parts and questions can have prerequisites. Adding prerequisites to questions is merely a matter of convenience though, as conceptually, one could do with only adding prerequisites to survey parts. A drawback of that approach, however, would be that a survey part would have to be created every time one would want to add a prerequisite to a single question.
There is nothing surprising about the structure of a question - it has some text, a number of possible answer options, a type (single choice, multiple choice, free text) and a data type (purely for the benefit of the researchers, as it might ease result processing).

\subsection{Survey Instance}
\label{subsec:surveyinstance}
The key idea behind differentiating between survey scheme and instance is, that a survey scheme could be stored centrally (e.g. on a server) and then distributed to all potential respondents, whereupon individual survey instances would be created on each individual smartphone. The individual user answers would be stored in a survey instance along with the users personal data and could then be sent back to some central repository. Distinguishing between multiple instances of a survey on an individual users smartphone would also be possible, thus conveniently enabling the survey recurrence feature.

\section{Design Considerations}
\label{sec:designconsiderations}
%TODO section intro
%TODO this section needs a lot more written in it
When running a survey on the phone, the user will expect the questions to be stored intermediately, allowing him/her to navigate back and forth between questions without discarding the answers already typed in.

\begin{figure}[!ht]
\centering
\begin{tikzpicture}[>=stealth',shorten >=1pt,auto,node distance=2.0cm]
\node[initial,state,label=$q_1$] (q1) {};
\node[state,label=above:$q_2$] (q2) [below left of=q1] {};
\node[state,label=above:$q_3$] (q3) [below right of=q1] {};
\node[state,label=above:$q_4$] (q4) [below left of=q2] {};
\path[->]
(q1) edge node {male} (q2)
(q1) edge node {female} (q3)
(q2) edge node {stuff} (q4);
\end{tikzpicture} \hspace{10mm}
\caption{CHANGE ME.}
\label{fig:answertree}
\end{figure}