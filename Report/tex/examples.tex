\chapter{System Examples}
\label{chap:examples}
In this chapter, examples of how the system might work are presented.

\section{System Components}
\label{sec:systemcomponents}
The final application will consist of two parts: 

\begin{description}
\item[Smartphone app] Used by the people who answer the surveys. In the application, available surveys are listed and can be filled out by the user. Additionally, the user can review surveys they already have answered.
\item[Desktop application] Used by the creators and editors of the surveys. In the application, new surveys can be created either by entering a text adhering to the DSL for survey definition or by drag-and-drop method using graphical components, which in turn are automatically translated to a DSL adherent text.
\end{description}

The figure <REF> shows how a screen where the user is able to select surveys to view and/or answer might look like.
%TODO We need 1 more mockup of the user application. At the moment, we only have a single question being answered screen - should add one listing available surveys/results

The figure \ref {fig:survey-app-mockup} shows how a display of a single question might look like when the user is in the process of filling out a survey.
\begin{figure}[!ht]
  \centering
  \includegraphics[scale=0.25]{\imgdir survey-app-mockup}
  \caption{User interface conceptual mock-up.}
  \label{fig:survey-app-mockup}
\end{figure}

The figure \ref{fig:survey-editor-mockup} shows how the survey editor might look like. It consists of a graphical toolbox panel, DSL adherent text panel and a view of the survey. The live view of the survey would be one of the key features, as it provides the users with instant feedback based on their actions and changes they make.

\begin{figure}[!ht]
  \centering
  \includegraphics[scale=0.6]{\imgdir survey-editor-mockup}
  \caption{Survey editor conceptual mock-up.}
  \label{fig:survey-editor-mockup}
\end{figure}

\section{Questionnaires}
\label{sec:questionnaires}
In this section, a couple example surveys are presented both in natural language and DSL adherent text.

\subsection{Health Questionnaire}
\label{subsec:healthquestionnaire}
The first survey is concerned with subjective health reporting on a weekly and daily basis. In this survey, the patient responds to a small questionnaire every day, and the survey is extended on a monthly basis to ask additional questions.\\\\
The first part of the survey gathers some personal information. Since this information is not ever going to change, it is reasonable to collect in only once and not every time the user answers the health survey. The questions in this part are all mandatory (a mandatory question must be answered before the user is allowed to proceed to next one).

\begin{enumerate}
\item ``What is your gender?''. A single answer from possibilities "Male/Female/Other" is expected.
\item ``What year were your born in?''. The user is expected to type in their year of birth into a textbox.
\item ``What country do you come from?''. The user is expected to type in their country of origin into a textbox.
\end{enumerate}

In the daily part of the survey, the following questions are asked: 

\begin{enumerate}
\item ``How are you feeling today?''. A mandatory question. The patient is expected to provide a single number on a scale from 1 to 10.
\item ``What is the reason of your discomfort?''. This question is asked, if the previous answer was below 5 (i.e. indicating that the patient is not feeling very well). The patient then must elaborate on the reason behind their discomfort. If the previous question indicated the patient was feeling well (5 and above), this one would simply not be displayed.
\item ``How would you describe your heat comfort?''. An optional question. A single answer from a list of possibilities is expected.
\item ``What is the ambient temperature?''. An optional question. The patient can either enter the temperature into a text box, or use the smartphone temperature sensor to automatically provide an answer.
\end{enumerate}

The last part of the survey concerns itself with allergies and is to be filled monthly. All questions are mandatory in this part.

\begin{enumerate}
\item ``Did you experience any allergies''. A Yes/No question.
\item ``Which of the following did cause an allergic reaction for you?''. This question is asked, if the previous answer was "Yes" and not displayed otherwise. The patient must elaborate on which allergens he/she reacted to. The question is of multiple choice type and also provides an "other" option, if the allergen was not in the list.
\item ``Please enter the other food(s) which caused an allergic reaction for you.''. This question is asked, if the "Other" answer was marked in the previous question and not displayed otherwise. The patient is expected to write the other allergens in a textbox.
\end{enumerate}

The DSL adherent representation of the survey outlined above looks as follows:
\begin{verbatim}
SURVEY Patient examination:

 PART Personal information:
  REPEAT: Once
   QUESTION 1: What is your gender?
    MANDATORY: Yes
    ANSWER TYPE: Single 
    ANSWERS: Male / Female / Other
      
   QUESTION 2: What year were you born in?
    MANDATORY: Yes
    ANSWER TYPE: Free

   QUESTION 3: What country do you come from?
    MANDATORY: Yes
    ANSWER TYPE: Free
      
 PART Comfort:
  REPEAT: Daily
   QUESTION 1: How are you feeling today?
   MANDATORY: Yes
   ANSWER TYPE: Single / Integer
   ANSWERS: 1..10
   
  QUESTION 2: What is the reason of your discomfort?
   MANDATORY: Yes
   PREREQUISITES: Q1 -> <5
   ANSWER TYPE: Free
 
  QUESTION 3: How would you describe your heat comfort?
   MANDATORY: No
   ANSWER TYPE: Single
   ANSWERS: Freezing / Cold / Acceptable / Warm / Hot
 
  QUESTION 4: What is the ambient temperature?
   MANDATORY: No
   ANSWER TYPE: Free / Temperature sensor
      
 PART Allergies:  
  REPEAT: Monthly
   QUESTION 1: Did you experience any allergic reactions recently?
    MANDATORY: Yes
    ANSWER TYPE: Single / Boolean
    ANSWERS: Yes / No
 
   QUESTION 2: Which of the following did cause an allergic reaction for you?
    MANDATORY: Yes
    PREREQUISITES: Q1 -> Yes
    ANSWER TYPE: Multiple / No limit
    ANSWERS: Nuts / Milk / Egg / Wheat / Soy / Fish / Other
 
   QUESTION 3: Please enter the other food(s) which caused an allergic reaction for you.
    MANDATORY: Yes
    PREREQUISITES: Q2 -> Other
    ANSWER TYPE: Free
\end{verbatim}

\subsection{Festival Questionnaire}
\label{subsec:festivalquestionnaire}
The second, short questionnaire is meant to measure the satisfaction of participants during and after a music festival. All the questions are mandatory, except for one using sensors and one asking for further comments.
\begin{enumerate}
\item ``What is the name of the band playing?''. The user has a choice to select one from all the musicians performing at the festival. This question is asked hourly.
\item ``Please rate the bands performance.'' The user is expected to provide a number on a scale from 1 to 10.
\item ``Please indicate your agreement with the following statements.''
\begin{enumerate}
\item The music is too loud (asked hourly, along with first question). If too loud, the user is asked to make a sound intensity measurement with the smartphone sensor.
\item The artist is too mainstream (asked hourly, along with first question)
\item The beer is too warm (asked daily)
\item The showers are too chilly (asked daily)
\item The toilets are too dirty (asked daily)
\end{enumerate} 
Each of the statements above is displayed as a single question, with following answer possibilities, of which the user is expected to pick one: ``agree'', ``agree'', ``not sure'', ``disagree'', ``strongly disagree''.
\item After the festival has ended, the user is asked to provide their overall liking of the festival and, optionally, extra comments.
\end{enumerate}

The DSL adherent text would look as follows:
\begin{verbatim}
SURVEY Festival satisfaction:

 PART Current artist:
  REPEAT: Hourly
   QUESTION 1: What is the name of the band playing?
    MANDATORY: Yes
    ANSWER TYPE: Free
      
   QUESTION 2: Please rate the bands performance.
    MANDATORY: Yes
    ANSWER TYPE: Single / Integer
    Answers: 1..10

   QUESTION 3: The music is too loud.
    MANDATORY: Yes
    ANSWER TYPE: Single
    ANSWERS: Strongly agree / Agree / Not sure / Disagree / Strongly disagree
    
   QUESTION 4: Please measure the sound intensity with your smartphone sound sensor.
    MANDATORY: No
    PREREQUISITES: Q3 -> Strongly agree OR Q3 -> Agree
    ANSWER TYPE: Sound sensor
    
   QUESTION 5: The artist is too mainstream.
    MANDATORY: Yes
    ANSWER TYPE: Single
    ANSWERS: Strongly agree / Agree / Not sure / Disagree / Strongly disagree
    
 PART Facilities:
  REPEAT: Daily
   QUESTION 1: The beer is too warm.
    MANDATORY: Yes
    ANSWER TYPE: Single
    ANSWERS: Strongly agree / Agree / Not sure / Disagree / Strongly disagree
    
   QUESTION 2: The showers are too chilly.
    MANDATORY: Yes
    ANSWER TYPE: Single
    ANSWERS: Strongly agree / Agree / Not sure / Disagree / Strongly disagree
    
   QUESTION 3: The toilets are too dirty.
    MANDATORY: Yes
    ANSWER TYPE: Single
    ANSWERS: Strongly agree / Agree / Not sure / Disagree / Strongly disagree
    
 PART Overall:
  REPEAT: Once
   QUESTION 1: How did you like the festival overall?
    MANDATORY: Yes
    ANSWER TYPE: Single
    ANSWERS: 1..10
    
   QUESTION 2: Are there any comments, ideas or suggestions you would like to pass on to us?
    MANDATORY: No
    ANSWER TYPE: Free
\end{verbatim}

The above survey examples illustrate the immense flexibility of the DSL based surveys: questions can be asked or skipped based on user answers to previous questions, they can be mandatory or optional, and parts of the survey can be repeated at arbitrary time intervals. These features create rich possibilities of designing a complex logical flow through a survey. It has also been demonstrated, that is possible to make use of built-in sensors in the smartphones, thereby extending the data collection toolset even further.