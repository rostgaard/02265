\chapter{System Examples}
\label{chap:examples}
The final application will consist of two parts: user mobile application and editor desktop application.
The user application will display available surveys. The user can pick them and fill in, using the interface presented in the mockup image on the left. User is also able to read the previous survey answers and results. 
\begin{figure}
  \centering
  \includegraphics[scale=0.25]{\imgdir survey-app-mockup}
  \label{fig:survey-app-mockup}
  \caption{User interface conceptual mock-up}
\end{figure}

The editor application allows creating a new survey. It contains three panels: toolbox/properties, source code and preview. It allows to build a survey by dragging and dropping its parts from a palette and also to define it by writing a text which adheres to a DSL.

\begin{figure}
  \centering
  \includegraphics[scale=0.6]{\imgdir survey-editor-mockup}
  \label{fig:survey-editor-mockup}
  \caption{Survey editor conceptual mock-up}
\end{figure}

A few example surveys are presented below.
\subsection{Health questionnaire}
%TODO add questionaire from DSL example.
The first survey is concerned with subjective health reporting on a weekly and daily basis. In this survey, the patient responds to a small questionnaire every day, and the survey is extended on a weekly basis to ask additional questions.\\\\
If a question is marked as ``mandatory'', it has to be answered before the next one is displayed.
In the first (daily) part, the following questions are asked:
A mandatory question: ``How are you feeling today''. The patient should choose a single number from 1 to 10.
Question only executed, if the 1.st question was answered with a value below 5. ``What is the reason of the discomfort (describe any factors which contributed to a low rating in the question number 1)'': The patient should answer in open text.
Question only executed, if the 1.st question was answered with a value below 5. ``Did you experience any of the following symptoms: Nausea, diarrhea, headache, other (open text)”. User should be able to choose any number of suggested answers and add an additional symptom, if not listed.
\begin{enumerate}
  \item A mandatory question: ``How would you describe your heat comfort: Freezing, cold, acceptable, warm, hot''. The surveyee answers by selecting one of the options
  \item A mandatory question. ``What is the ambient temperature'': The user enters the number by hand using the preferred temperature scale (Fahrenheit, Celsius) or uses the built-in sensor to enter it.
\end{enumerate}
\noindent
The second part of the survey is to be filled monthly, on every:

\begin{enumerate}

\item A mandatory question: ``Did you experience any allergies''. A Yes/No question.
\item Conditional question: if the 1.st question was answered with ``Yes'': ``Did have allergy for: (multiple) nuts, water, chocolate, milk, other (multiple open text)''. Multiple options can be chosen, and if the answer does not appear on the list, a user can enter a new one.
\item Question only executed, if the 1.st question was answered with ``Yes'': ``How strong was the allergy from 0\% to 100\%, based on your personal experience of allergies''. User can select any number in a continuous way (e.g. with a slider).
\end{enumerate}

\subsection{Festival questionnaire}
The second questionnaire is meant to measure the satisfaction of participants in the rock music festival. To gather enough data, a user is asked to answer 10 times, or by June 19th 2015 the latest.
\begin{enumerate}
\item Mandatory question: ``What is the name of the band playing''. User has a choice to select one from all the musicians performing on the concert.
\item Mandatory question. ``Do you agree on any of the following statements: The music was too loud, the artist was too mainstream the beer was too warm, the showers were too chilly, the toilets were too dirty''. The user gives each answer a rating, selecting between ``Agree'', ``slightly agree'', ``not sure'', ``slightly disagree'', ``strongly disagree''.
\end{enumerate}

\subsection{Kindergarten questionnaire}
Nowadays even toddlers have smartphones! Let's ask how do they feel about their life in the kindergarten and use the information to improve the conditions in which they live.
\begin{enumerate}
  \item Mandatory question. ``Are you boy or girl?''. Selection of one from the two options
  \item Mandatory question. ``How old are you?''. Pick up a number between 3 and 7, representing the age of the questionee.
  \item Question only executed, if the 1.st question was answered with ``Boy'' and the 2nd one had age of 6,7 selected.. ``Did you play with cars?''. A yes/no question.
  \item Question only executed, if the 1.st question was answered with ``Girl'' and the 2nd one had age of 5,6,7 selected.. ``Did you play with dolls?''. A yes/no question.''
  \item Mandatory question. ``How did you like the soup: ``Very much, It was good, I liked it, I don't like the soup, It was horrible''. One answer is chosen
  \item Mandatory question: ``Do you eat crayons?'', A Yes/No question
  \item Mandatory question, only executed if the question 6. was answered with ``No'': ``What else would you like to eat''. Open text answer.
\end{enumerate}
