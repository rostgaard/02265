\chapter{System Examples}
\label{chap:examples}
In this chapter, examples of how the system might work are presented.

\section{System Components}
\label{sec:systemcomponents}
The final application will consist of two parts: 

\begin{description}
\item[Smartphone app] Used by the people who answer the surveys. In the application, available surveys are listed and can be filled out by the user. Additionally, the user can review surveys they already have answered.
\item[Desktop application] Used by the creators and editors of the surveys. In the application, new surveys can be created either by entering a text adhering to the DSL for survey definition or by drag-and-drop method using graphical components, which in turn are automatically translated to a DSL adherent text.
\end{description}

The figure <REF> shows how a screen where the user is able to select surveys to view and/or answer might look like.
%TODO We need 1 more mockup of the user application. At the moment, we only have a single question being answered screen - should add one listing available surveys/results

The figure \ref {fig:survey-app-mockup} shows how a display of a single question might look like when the user is in the process of filling out a survey.
\begin{figure}[!ht]
  \centering
  \includegraphics[scale=0.25]{\imgdir survey-app-mockup}
  \caption{User interface conceptual mock-up.}
  \label{fig:survey-app-mockup}
\end{figure}

The figure \ref{fig:survey-editor-mockup} shows how the survey editor might look like. It consists of a graphical toolbox panel, DSL adherent text panel and a view of the survey. The live view of the survey would be one of the key features, as it provides the users with instant feedback based on their actions and changes they make.

\begin{figure}[!ht]
  \centering
  \includegraphics[scale=0.6]{\imgdir survey-editor-mockup}
  \caption{Survey editor conceptual mock-up.}
  \label{fig:survey-editor-mockup}
\end{figure}

\section{Questionnaires}
\label{sec:questionnaires}
In this section, a few example surveys are presented. The examples are shown both in natural language and DSL adherent text.

\subsection{Health Questionnaire}
\label{subsec:healthquestionnaire}
The first survey is concerned with subjective health reporting on a weekly and daily basis. In this survey, the patient responds to a small questionnaire every day, and the survey is extended on a monthly basis to ask additional questions.\\\\
The first part of the survey gathers some personal information. Since this information is not ever going to change, it is reasonable to collect in only once and not every time the user answers the health survey. The questions in this part are all mandatory (a mandatory question must be answered before the user is allowed to proceed to next one).

\begin{enumerate}
\item ``What is your gender?''. A single answer from possibilities "Male/Female/Other" is expected.
\item ``What year were your born in?''. The user is expected to type in their year of birth into a textbox.
\item ``What country do you come from?''. The user is expected to type in their country of origin into a textbox.
\end{enumerate}

In the daily part of the survey, the following questions are asked: 

\begin{enumerate}
\item ``How are you feeling today?''. A mandatory question. The patient is expected to provide a single number on a scale from 1 to 10.
\item ``What is the reason of your discomfort?''. This question is asked, if the previous answer was below 5 (i.e. indicating that the patient is not feeling very well). The patient then must elaborate on the reason behind their discomfort. If the previous question indicated the patient was feeling well (5 and above), this one would simply not be displayed.
\item ``How would you describe your heat comfort?''. An optional question. A single answer from a list of possibilities is expected.
\item ``What is the ambient temperature?''. An optional question. The patient can either enter the temperature into a text box, or use the smartphone temperature sensor to automatically provide an answer.
\end{enumerate}

The last part of the survey concerns itself with allergies and is to be filled monthly. All questions are mandatory in this part.

\begin{enumerate}
\item ``Did you experience any allergies''. A Yes/No question.
\item ``Which of the following did cause an allergic reaction for you?''. This question is asked, if the previous answer was "Yes" and not displayed otherwise. The patient must elaborate on which allergens he/she reacted to. The question is of multiple choice type and also provides an "other" option, if the allergen was not in the list.
\item ``Please enter the other food(s) which caused an allergic reaction for you.''. This question is asked, if the "Other" answer was marked in the previous question and not displayed otherwise. The patient is expected to write the other allergens in a textbox.
\end{enumerate}

The DSL adherent representation of the survey outlined above looks as follows:
\begin{verbatim}
DSL Example (as being entered when first creating a survey)

SURVEY Patient examination:

 PART Personal information:
  REPEAT: Once
   QUESTION 1: What is your gender?
    MANDATORY: Yes
    ANSWER TYPE: Single 
    ANSWERS: Male / Female / Other
      
   QUESTION 2: What year were you born in?
    MANDATORY: Yes
    ANSWER TYPE: Free

   QUESTION 3: What country do you come from?
    MANDATORY: Yes
    ANSWER TYPE: Free
      
 PART Comfort:
  REPEAT: Daily
   QUESTION 1: How are you feeling today?
   MANDATORY: Yes
   ANSWER TYPE: Single / Integer
   ANSWERS: 1..10
   
  QUESTION 2: What is the reason of your discomfort?
   MANDATORY: Yes
   PREREQUISITES: Q1 -> <5
   ANSWER TYPE: Free
 
  QUESTION 3: How would you describe your heat comfort?
   MANDATORY: No
   ANSWER TYPE: Single
   ANSWERS: Freezing / Cold / Acceptable / Warm / Hot
 
  QUESTION 4: What is the ambient temperature?
   MANDATORY: No
   ANSWER TYPE: Free / Temperature sensor
      
 PART Allergies:  
  REPEAT: Monthly
   QUESTION 1: Did you experience any allergic reactions recently?
    MANDATORY: Yes
    ANSWER TYPE: Single / Boolean
    ANSWERS: Yes / No
 
   QUESTION 2: Which of the following did cause an allergic reaction for you?
    MANDATORY: Yes
    PREREQUISITES: Q1 -> Yes
    ANSWER TYPE: Multiple / No limit
    ANSWERS: Nuts / Milk / Egg / Wheat / Soy / Fish / Other
 
   QUESTION 3: Please enter the other food(s) which caused an allergic reaction for you.
    MANDATORY: Yes
    PREREQUISITES: Q2 -> Other
    ANSWER TYPE: Free
\end{verbatim}

\subsection{Festival Questionnaire}
\label{subsec:festivalquestionnaire}
The second questionnaire is meant to measure the satisfaction of participants in the rock music festival. To gather enough data, a user is asked to answer 10 times, or by June 19th 2015 the latest.
\begin{enumerate}
\item Mandatory question: ``What is the name of the band playing''. User has a choice to select one from all the musicians performing on the concert.
\item Mandatory question. ``Do you agree on any of the following statements: The music was too loud, the artist was too mainstream the beer was too warm, the showers were too chilly, the toilets were too dirty''. The user gives each answer a rating, selecting between ``Agree'', ``slightly agree'', ``not sure'', ``slightly disagree'', ``strongly disagree''.
\end{enumerate}

\subsection{Kindergarten Questionnaire}
\label{subsec:kindergartenquestionnaire}
Nowadays, even toddlers have smartphones! Let's ask how do they feel about their life in the kindergarten and use the information to improve the conditions in which they live.
\begin{enumerate}
  \item Mandatory question. ``Are you boy or girl?''. Selection of one from the two options
  \item Mandatory question. ``How old are you?''. Pick up a number between 3 and 7, representing the age of the questionee.
  \item Question only executed, if the 1.st question was answered with ``Boy'' and the 2nd one had age of 6,7 selected.. ``Did you play with cars?''. A yes/no question.
  \item Question only executed, if the 1.st question was answered with ``Girl'' and the 2nd one had age of 5,6,7 selected.. ``Did you play with dolls?''. A yes/no question.''
  \item Mandatory question. ``How did you like the soup: ``Very much, It was good, I liked it, I don't like the soup, It was horrible''. One answer is chosen
  \item Mandatory question: ``Do you eat crayons?'', A Yes/No question
  \item Mandatory question, only executed if the question 6. was answered with ``No'': ``What else would you like to eat''. Open text answer.
\end{enumerate}
