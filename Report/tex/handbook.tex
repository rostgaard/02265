\chapter{Handbook}
\label{chap:handbook}
This chapter provides instructions for using the system prototype through a series of screenshots and explanatory text.\\\\
For easy installation of the app on an Android device, an .apk file is provided. In this handbook, it is assumed that the app is already installed and started on the smartphone, and displays the initial screen shown in figure \ref{fig:screen-initial-page}. Pressing the button ``Filled surveys'' will send the user to the screen shown in figure \ref{fig:screen-list-surveys-none}, where he/she will be able to see a list of previously filled out surveys. This list is empty for now. Going back (back arrow near the top of the screen) will send the user to the initial screen where he/she can press ``Surveys to fill'' to see a list of pending surveys.\\\\
\begin{figure}[!htbp]
 \centering

 \begin{subfigure}{.3\textwidth}
    \centering
    \includegraphics[width=.95\linewidth]{\imgdir screen-initial-page}
    \caption{Initial screen}
    \label{fig:screen-initial-page}
  \end{subfigure}
  \begin{subfigure}{.3\textwidth}
    \centering
    \includegraphics[width=.95\linewidth]{\imgdir screen-list-surveys-none}
    \caption{Saved survey list}
    \label{fig:screen-list-surveys-none}
  \end{subfigure}

  \caption{The first screens that the user is presented with}
  \label{fig:screen-initial-pages}
\end{figure}

The next screen the user is presented with after pressing ``Surveys to fill'', is the list of pending surveys as seen in figure \ref{fig:screen-pending-surveys}. A pending survey may be selected by pressing it.
\newpage

\begin{figure}[!htbp]
 \centering
 \begin{subfigure}{.3\textwidth}
    \centering
    \includegraphics[width=.95\linewidth]{\imgdir screen-pending-surveys}
    \caption{List of pending surveys}
    \label{fig:screen-pending-surveys}
  \end{subfigure}
  \begin{subfigure}{.3\textwidth}
    \centering
    \includegraphics[width=.95\linewidth]{\imgdir screen-pending-surveys-pressed}
    \caption{Selecting an item}
    \label{fig:screen-pending-surveys-pressed}
  \end{subfigure}
  \caption{Selecting a pending survey}
   \label{fig:screens-selecting-pending-survey}
\end{figure}

After a survey has been selected from the pending list, it will prompt the user the first question.

\begin{figure}[!htbp]
  \centering
  \begin{subfigure}{.3\textwidth}
    \centering
    \includegraphics[width=.95\linewidth]{\imgdir screen-question-1}
    \caption{First question}
    \label{fig:screen-question-1}
  \end{subfigure}
  \begin{subfigure}{.3\textwidth}
    \centering
    \includegraphics[width=.95\linewidth]{\imgdir screen-question-1-answer}
    \caption{Input the answer}
    \label{fig:screen-question-1-answer}
  \end{subfigure}
    \begin{subfigure}{.3\textwidth}
    \centering
    \includegraphics[width=.95\linewidth]{\imgdir screen-question-1-answer-selected}
    \caption{Selecting the answer}
    \label{fig:screen-question-1-answer-selected}
  \end{subfigure}
  \caption{Answering the first question}
  \label{fig:screen-first-question}
\end{figure}

The question illustrated in figure \ref{fig:screen-first-question} is about the gender of the person answering the survey. An answer may be input by pressing the blank space below the question. The user interface chooses different input methods, based how the question should be answered. Figure \ref{fig:screen-question-1-answer} and \ref{fig:screen-question-1-answer-selected} illustrates that the input method for this question is a single value from a predefined list of answers. The user may scroll up and down to select value the correct gender and press ``OK'' to accept the input. The ``Next'' button may now be pressed in order to continue to the next question.

\begin{figure}[!htbp]
  \centering
  \begin{subfigure}{.3\textwidth}
    \centering
    \includegraphics[width=.95\linewidth]{\imgdir screen-question-2}
    \caption{Second question}
    \label{fig:screen-question-2}
  \end{subfigure}
  \begin{subfigure}{.3\textwidth}
    \centering
    \includegraphics[width=.95\linewidth]{\imgdir screen-question-2-answer}
    \caption{Input the answer}
    \label{fig:screen-question-2-answer}
  \end{subfigure}
  \begin{subfigure}{.3\textwidth}
    \centering
    \includegraphics[width=.95\linewidth]{\imgdir screen-question-2-answer-selected}
    \caption{Selecting the answer}
    \label{fig:screen-question-2-answer-selected}
  \end{subfigure}
  \caption{Answering the second question}
  \label{fig:screen-second-question}
\end{figure}

Second question (figure \ref{fig:screen-second-question}), is about the year of birth. Following the same procedure on inputting answers as in the first question, it can be seen that there is a slight difference in the way the answer is input. Where the first question only allowed predefined input, this question now allows numeric input. Another thing to note, is that the ``Previous'' button is now pressable, so that the user may go back and change the answer on the previous question.
\newpage

\begin{figure}[!htbp]
  \centering
  \begin{subfigure}{.3\textwidth}
    \centering
    \includegraphics[width=.95\linewidth]{\imgdir screen-question-3}
    \caption{Third question}
    \label{fig:screen-question-3}
  \end{subfigure}
  \begin{subfigure}{.3\textwidth}
    \centering
    \includegraphics[width=.95\linewidth]{\imgdir screen-question-3-answer}
    \caption{Answer to third question}
    \label{fig:screen-question-3-answer}
  \end{subfigure}
  \caption{Answering the third question}
  \label{fig:screen-third-question}
\end{figure}

The third question (figure \ref{fig:screen-third-question}) is similar to the first question, but allows input within a numeric range of 1 upto 10 indicating the level of comfort of the user. Once again, the user may go back and change previous answers by pressing the ``Previous'' button.
\newpage

\begin{figure}[!htbp]
  \centering
  \begin{subfigure}{.3\textwidth}
    \centering
    \includegraphics[width=.95\linewidth]{\imgdir screen-question-4-answer}
    \caption{Fourth question - single answer}
    \label{fig:screen-question-4-answer}
  \end{subfigure}
  \begin{subfigure}{.3\textwidth}
    \centering
    \includegraphics[width=.95\linewidth]{\imgdir screen-question-5-answer}
    \caption{Fifth question - multiple answers}
    \label{fig:screen-question-5-answer}
  \end{subfigure}
  \caption{The fourth and fifth question}
  \label{fig:screen-fourth-and-fifth-question}
\end{figure}

The fourth and fifth questions (figure \ref{fig:screen-fourth-and-fifth-question}) are related to each other in the way that fifth question is only answered if the fourth one is answered with a ``yes''. The fifth question shows an example of a question that may have several answers. The answers are predefined set of answers values.
\newpage

When the survey is answered, the user is presented with a confirmation ans show in figure \ref{fig:screen-thank-you}. The user may then choose to either go back and change the answers (figure \ref{fig:screen-thank-you-change-answer}), or simply just submit the survey (figure \ref{fig:screen-thank-you-submit}).

\begin{figure}[!htbp]
  \centering
  \begin{subfigure}{.3\textwidth}
    \centering
    \includegraphics[width=.95\linewidth]{\imgdir screen-thank-you}
    \caption{Confirmation of filled out survey}
    \label{fig:screen-thank-you}
  \end{subfigure}
  \begin{subfigure}{.3\textwidth}
    \centering
    \includegraphics[width=.95\linewidth]{\imgdir screen-thank-you-change-answer}
    \caption{Go back and change the answer}
    \label{fig:screen-thank-you-change-answer}
  \end{subfigure}
  \begin{subfigure}{.3\textwidth}
    \centering
    \includegraphics[width=.95\linewidth]{\imgdir screen-thank-you-submit}
    \caption{Submit the answer to server}
    \label{fig:screen-thank-you-submit}
  \end{subfigure}
  \caption{Available options after filling in the survey}
  \label{fig:screen-post-survey-action}
\end{figure}

Once the survey is submitted, the is returned to the main menu and may start another pending survey, or view previously answered surveys.
\newpage

\begin{figure}[!htbp]
  \centering
  \begin{subfigure}{.3\textwidth}
    \centering
    \includegraphics[width=.95\linewidth]{\imgdir screen-initial-page}
    \caption{Back at initial screen, choose ``Filled surveys''}
    \label{fig:screen-initial-page-again}
  \end{subfigure}
  \begin{subfigure}{.3\textwidth}
    \centering
    \includegraphics[width=.95\linewidth]{\imgdir screen-list-surveys-one}
    \caption{List of saved surveys now includes out answer}
    \label{fig:screen-list-surveys-one}
  \end{subfigure}
  \begin{subfigure}{.3\textwidth}
    \centering
    \includegraphics[width=.95\linewidth]{\imgdir screen-display-survey}
    \caption{Answers to the previously filled in survey}
    \label{fig:screen-display-survey}
  \end{subfigure}
  \caption{Displaying the answers to the previously filled in survey}
  \label{fig:screen-display-filled-in-survey}
\end{figure}

In the completed survey list, the user can press on any survey in the list and its result will be displayed (currently merely as JSON text; conceivably in a different, more user-friendly way in the final system).