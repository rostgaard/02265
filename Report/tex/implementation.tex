\chapter{Implementation}
\label{chap:implementation}
This chapter provides information about implementation of the system. Specifically, the created domain specific language is presented and some details of the smartphone app prototype are discussed.

\section{Domain Specific Language}
\label{sec:dsl}
Surveys are to be specified by writing text adherent to the DSLs grammar. The eBNF grammar definition for the DSL is shown below. It reflects the scheme structure shown in the metamodel, in figure \ref{fig:metamodel}. Examples of DSL adherent text for survey definition have been presented in section \ref{sec:questionnaires}.

\begin{grammar}
      <survey> ::= `SURVEY', <text>, `:', <parts> ;
      
      <parts> ::= <part>, $\lbrace$ <part> $\rbrace$ ; 
       
      <part> ::= `PART', <text>, `:', {[} <repetitions> {]}, <questions> ;
      
      <repetitions> ::= `REPEAT:', <frequency> ;
      
      <questions> ::= <question>, $\lbrace$ <question> $\rbrace$ ;
      
      <question> ::= `QUESTION', <number>, `:', <text>, <options>, <answers> ;
      
      <options> ::= {[} <mandatory> {]}, {[} <prerequisites> {]}, <answer type> ;
      
      <mandatory> ::= `MANDATORY:', <bool> ;
      
      <prerequisites> ::= `PREREQUISITES:', ( $\lbrace$ <prerequisite> $\rbrace$ | `None' ) ;
      
      <answer type> ::= `ANSWER TYPE:', <choice> | <free> ;
      
      <choice> ::= <multiplicity>, {[} `/', <type> {]} ;
      
      <free> ::= `Free', {[} `/', <sensor> {]} ;
      
      <prerequisite> ::= `Q', <number>, `->', ( <answer> | <expression> ) ;
      
      <answers> ::= `ANSWERS:', <answer>, `/', <answer>, $\lbrace$ `/', <answer> $\rbrace$ ;
      
      <answer> ::= <number>, <text> ;
         
      <multiplicity> ::= `Single' | ( `Multiple', {[} `/', <limit> {]} ) ;
      
      <frequency> ::= `Hourly' | `Daily' | `Weekly' | `Bi-Weekly' | `Monthly' | `Yearly' | <other> ;
      
      <limit> ::= ( <number>, `..', <number> ) | `No limit' ;
      
      <sensor> ::= ( `Temperature' | `GPS' | <other> ), `sensor' ;
      
      <bool> ::= `Yes' | `No' ;
      
      <type> ::= `Integer' | `Float' | `Boolean' | `String' | `Date' | <other> ;
      
      <expression> ::= ?logic expression? ;
      
      <text> ::= ?unicode literal? ;
      
      <number> ::= {[} `-' {]}, ?numeric literal? ;
      
      <other> ::= ?future addition? ;
\end{grammar}

\section{Smartphone App}
\label{sec:app}
In this section, the more interesting details of the smartphone app prototype implementation are outlined.

\subsection{App Workflow}
\label{subsec:workflow}
The workflow of application at the start is depicted in figure \ref{fig:startup-workflow}. At the startup of the application, the backend checks, if any new schema appeared and if yes, generates new instance for each new schema, checks which instances should be active and adds them to display (available) list. When done, it serializes to file all new instances and launches the user interface using the display list. 

\begin{figure}[!htbp]
  \centering
    \centering
    \includegraphics[scale=0.5]{\imgdir workflow_atise}
    \caption{Logic of the application in the start up (where start - start of the application and end - launching user interface)}
    \label{fig:startup-workflow}
\end{figure}

\subsection{Cross-Platform}
\label{subsec:crossplatform}
The app has been implemented using Xamarin framework - a tool, which allows for easy development of cross-platform programs. Currently, the app is available only for Android phones; this is mostly caused by high difficulty in installing 3rd party apps on iPhones. However, none of the functionalities have been implemented using platform-specific libraries, so it would be possible to easily port the app after obtaining appropriate developer licenses. 

\subsection{Serialization}
\label{subsec:serialization}
Serialization is an important part of the application. Desirably, all surveys are to be downloaded from the cloud, so having a light-weight and standardized intermediate format is very helpful. Additionally, the app needs to save the answered surveys and eventually also unfinished surveys.

A decision was made to use JSON as the intermediate format and Json.NET as a serialization framework. It allows easy serialization and deserialization of .NET classes, with respect to heritage and interfaces. 
\newpage

\subsection{Current State}
\label{subsec:current state}
Currently, the app supports following features:
\begin{itemize}
\item displaying list of available surveys
\item filling in the surveys and saving it
\item browsing previously answered surveys
\item scheduling surveys
\item displaying questions depending on previous answers
\end{itemize}

The features, which are yet to be implemented are:
\begin{itemize}
\item synchronizing with server
\item saving half-filled surveys
\item better interface for browsing answered surveys
\end{itemize}

More future work possibilities are outlined in section \ref{sec:futurework}.